\documentclass[DIN, pagenumber=false, fontsize=11pt, parskip=half]{scrartcl}

\usepackage[utf8]{inputenc}
\usepackage[T1]{fontenc}
\usepackage{textcomp}
\usepackage{listings}
\usepackage{graphicx}
\usepackage{amsmath}
\usepackage{amssymb}

\setlength{\parindent}{0em}

% set section in CM
\setkomafont{section}{\normalfont\bfseries\Large}

\renewcommand{\title}[1]{{\noindent\Large\textbf{#1}}}


%===================================
\begin{document}

\noindent\textbf{Algorithms for Datascience} \hfill \textbf{Universität Bonn}\\
WS23/24 \hfill \\
Lea Silberberg, Jennifer Kroppen, Ellen Steffes, Elif Yildirir, Felix Lehmann

\title{Assignment Sheet 1 \hfill \today}

%===================================
\section*{Task 1}

Each item can be in exactly on of three places for a association rule,
\begin{enumerate}
    \item left hand side,
    \item right hand side,
    \item not part of the rule.
\end{enumerate}

Therefore this gives an upper limit for the number of rules to be $3^d$.

From these we have to subtract nonsensical cases where one of the side of the left or right hand side are emtpy.
This are $2 \times 2^d = 2^{d+1}$ cases.

The case where both sides are emtpy has now been subtraced twice so we have to add back $1$.

Levaing us with the formula:
\begin{equation*}
    3^d - 2^{d+1} + 1
\end{equation*}

%===================================
\section*{Task 2}

We found the following itemsets with a minimum frequency of $3$:\\
\begin{table}[h!]
    \begin{tabular}{ll}
        E   & 4 \\
        O   & 3 \\
        M   & 3 \\
        Y   & 3 \\
        K   & 5 \\
        EO  & 3 \\
        EK  & 4 \\
        KO  & 3 \\
        KM  & 3 \\
        KY  & 3 \\
        EKO & 3
    \end{tabular}
\end{table}


%===================================
\section*{Task 3}

\end{document}